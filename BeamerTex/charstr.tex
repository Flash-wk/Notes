%\begin{frame}[fragile]
%\frametitle{Storage of Characters}
%\begin{columns}
%\begin{column}{0.45\textwidth}
%\begin{itemize}
%\end{itemize}
%\end{column}
%\begin{column}{0.45\textwidth}
%\end{column}
%\end{columns}
%\end{frame}

\section{I : Characters \& Strings}
\label{chap:charstr}

\begin{frame}[fragile]
\frametitle{Storage of Characters}
\begin{columns}

\begin{column}{0.45\textwidth}
\begin{itemize}
\item Characters are stored in the machine as one byte (generally $8$-bits storing one
of {\bf 256} possible values).
\item These may be thought of a characters, or very small integers.
\item Only a subset of these 256 values are required
for the printable characters, space, newline etc.
\item Declaration:
\begin{verbatim}
char c;

c = 'A';
\end{verbatim}
or :
\begin{verbatim}
char c1 = 'A', c2 = '*', c3 = ';' ;
\end{verbatim}
\end{itemize}
\end{column}

\begin{column}{0.45\textwidth}
\begin{itemize}
\item The particular integer used to represent a character
is dependent on the encoding used. The most common
of these, used on most UNIX and PC platforms, is ASCII.

\begin{center}
\begin{tabular}{|l|lllll|}\hline
lowercase   & 'a' & 'b' & 'c' & \verb^...^ & 'z' \\
ASCII value & 97  &  98 &  99 & \verb^...^ & 112 \\ \hline
uppercase   & 'A' & 'B' & 'C' & \verb^...^ & 'Z' \\
ASCII value & 65  &  66 &  67 & \verb^...^ &  90 \\ \hline
digit       & '0' & '1' & '2' & \verb^...^ & '9' \\
ASCII value & 48  &  49 &  50 & \verb^...^ &  57 \\ \hline
other       & '\&' & '*' & '+' & \verb^...^ & \\
ASCII value & 38  &  42 &  43 & \verb^...^ & \\ \hline
\end{tabular}
\end{center}
\end{itemize}
\end{column}

\end{columns}
\end{frame}


%%%%%%%%%%%%%%%%%%%%%%%%%%%%%%%%%%%%%%%%%%%%%%%%%%%%%%%%%%%%%%%%%%


\begin{frame}[fragile]
\frametitle{Using Characters}
\begin{columns}

\begin{column}{0.45\textwidth}
\begin{itemize}
\item When using \verb^printf()^ and \verb^scanf()^ the formats
\verb^%c^ and \verb^%d^ do very different things :
\begin{verbatim}
char c = 'a'
printf("%c\n", c); /* prints : a  */
printf("%d\n", c); /* prints : 97 */
\end{verbatim}
\item Hard-to-print characters have an escape sequence
i.e. to print a newline, the 2 character escape \verb^'\n'^ is used.
\end{itemize}
\end{column}

\begin{column}{0.55\textwidth}
{\small
\begin{tabular}{|l|l|l|}\hline
Escape sequence	& Hex value & Character \\ \hline
\textbackslash a & 07 & Alert (Beep, Bell)\\
\textbackslash b &08&Backspace\\
\textbackslash e &1B&Escape character\\
\textbackslash f &0C&Formfeed Page Break\\
\textbackslash n &0A&Newline (Line Feed)\\
\textbackslash r &0D&Carriage Return\\
\textbackslash t &09&Horizontal Tab\\
\textbackslash v &0B&Vertical Tab\\
\textbackslash\textbackslash &5C&Backslash\\
\textbackslash ' &27&Apostrophe \\
\textbackslash " &22&Double quote \\
\textbackslash ? &3F&Question mark \\ \hline
\end{tabular}
}
\end{column}

\end{columns}
\end{frame}


%%%%%%%%%%%%%%%%%%%%%%%%%%%%%%%%%%%%%%%%%%%%%%%%%%%%%%%%%%%%%%%%%%


\begin{frame}[fragile]
\frametitle{Using {\tt getchar()} and {\tt putchar()}}
\begin{columns}[T]
\begin{column}{0.45\textwidth}

\lstinputlisting[style=basicc]{../Code/ChapI/dblout1.c}
\outputlisting{../Code/ChapI/dblout1.manout}

{\footnotesize
This has the unfortunate problem of requiring a `special' character to terminate.
More aggressively, the user could terminate by pressing {\tt CTRL-C}.
}
\end{column}

\begin{column}{0.45\textwidth}

\lstinputlisting[style=basicc]{../Code/ChapI/dblout2.c}
\outputlisting{../Code/ChapI/dblout2.manout}

{\footnotesize
The end-of-file constant is defined in {\tt stdio.h}. Although
system dependent, {\tt -1} is often used. On the UNIX system
this is generated when the end of a file being piped is reached,
or when {\tt CTRL-D} is pressed.
}
\end{column}
\end{columns}
\end{frame}


%%%%%%%%%%%%%%%%%%%%%%%%%%%%%%%%%%%%%%%%%%%%%%%%%%%%%%%%%%%%%%%%%%

\begin{frame}[fragile]
\frametitle{Capitalization}
\begin{columns}[T]
\begin{column}{0.45\textwidth}

\lstinputlisting[style=basicc]{../Code/ChapI/caps1.c}
\outputlisting{../Code/ChapI/caps1.manout}

{\footnotesize
This is more easily achieved by using some of
the definitions found in {\tt ctype.h}.
}
\end{column}

\begin{column}{0.45\textwidth}
\begin{center}
\begin{tabular}{|l|l|} \hline
Macro        & {\tt true} returned if: \\ \hline
isalnum(int c) & Letter or digit \\
isalpha(int c) & Letter \\
iscntrl(int c) & Control character \\
isdigit(int c) & Digit \\
isgraph(int c) & Printable (not space) \\
islower(int c) & Lowercase\\
isprint(int c) & Printable\\
ispunct(int c) & Punctuation\\
isspace(int c) & White Space\\
isupper(int c) & Uppercase\\
isxdigit(int c)& Hexadecimal\\
isascii(int c) & ASCII code \\ \hline
\end{tabular}
\end{center}
\end{column}

\end{columns}
\end{frame}

%%%%%%%%%%%%%%%%%%%%%%%%%%%%%%%%%%%%%%%%%%%%%%%%%%%%%%%%%%%%%%%%%%


\begin{frame}[fragile]
\frametitle{{\tt ctype.h}}
\begin{columns}[T]

\begin{column}{0.45\textwidth}
Some useful functions are :
\begin{center}
\begin{tabular}{|l|l|} \hline
Function/Macro & Returns: \\ \hline
int tolower(int c) & Lowercase c \\
int toupper(int c) & Uppercase c \\
int toascii(int c) & ASCII code for c \\ \hline
\end{tabular}
\end{center}
\lstinputlisting[style=basicc]{../Code/ChapI/caps2.c}
\end{column}


\begin{column}{0.45\textwidth}
\lstinputlisting[style=basicc]{../Code/ChapI/caps3.c}
\outputlisting{../Code/ChapI/caps3.manout}
\end{column}

\end{columns}
\end{frame}

%%%%%%%%%%%%%%%%%%%%%%%%%%%%%%%%%%%%%%%%%%%%%%%%%%%%%%%%%%%%%%%%%%

\begin{frame}[fragile]
\frametitle{Strings}
\begin{columns}
\begin{column}{0.45\textwidth}
\begin{itemize}
\item Strings are 1D arrays of characters.
\item Any character in a string may be accessed as an array
element.
\item The important difference between strings and ordinary arrays
is the {\bf end-of-string sentinel} \verb^'\0'^ or null character.
\item The string "abc" has a {\it length} of 3, but its {\it size} is 4.
\item Note \verb^'a'^ and \verb^"a"^ are different. The first is a
character constant, the second is a string with 2 elements
\verb^'a'^ and \verb^'\0'^.
\end{itemize}
\end{column}

\begin{column}{0.45\textwidth}
Initialising Strings~:
\begin{itemize}
\item \verb^char w[6] = "Hello";^
\item \begin{verbatim}
char w[250];
w[0] = 'a';
w[1] = 'b';
w[2] = 'c';
w[3] = '\0';
\end{verbatim}
\item \begin{verbatim}
scanf("%s", w);
\end{verbatim}
Removes leading spaces, reads a string (terminated by a
space or \verb^EOF^). Adds a null character to the end
of the string.
\item \begin{verbatim}
char w[250] = {'a', 'b', 'c', '\0'};
\end{verbatim}
\end{itemize}
\end{column}

\end{columns}
\end{frame}

%%%%%%%%%%%%%%%%%%%%%%%%%%%%%%%%%%%%%%%%%%%%%%%%%%%%%%%%%%%%%%%%%%


\begin{frame}[fragile]
\frametitle{Unused Letters and {\tt string.h}}
\begin{columns}

\begin{column}{0.45\textwidth}
\lstinputlisting[style=basicc]{../Code/ChapI/lazydog.c}
\outputlisting{../Code/ChapI/lazydog.autoout}
\end{column}

\begin{column}{0.45\textwidth}
In \verb^#include <string.h>^ :
{\small
\begin{verbatim}
char *strcat(char *dst, const char *src);
int strcmp(const char *s1, const char *s2);
\end{verbatim}
}
\begin{itemize}
\item \verb^strcat()^ appends a copy of string \verb^src^,
including  the  terminating null character,
to  the  end  of  string  \verb^dst^.
\item \verb^strcmp()^ compares two strings byte-by-byte, according to the
     ordering  of  your  machine's  character  set.  The function
     returns an integer greater than, equal to, or less  than  0,
     if the string pointed to by \verb^s1^ is greater than, equal to, or
     less than the string pointed to by \verb^s2^ respectively.
\end{itemize}
\end{column}

\end{columns}
\end{frame}


%%%%%%%%%%%%%%%%%%%%%%%%%%%%%%%%%%%%%%%%%%%%%%%%%%%%%%%%%%%%%%%%%%

\begin{frame}[fragile]
\frametitle{More {\tt string.h}}
\begin{columns}

\begin{column}{0.45\textwidth}
In \verb^#include <string.h>^ :
{\small
\begin{verbatim}
char *strcpy(char *dst, const char *src);
unsigned strlen(const char *s);
\end{verbatim}
}
\begin{itemize}
\item \verb^strcpy()^ copies string \verb^src^ to \verb^dst^
including the  terminating null  character,
stopping after the null character has been copied.
\item \verb^strlen()^ returns the number of bytes in \verb^s^,
not including the terminating null character.
\end{itemize}
\end{column}

\begin{column}{0.45\textwidth}
One way to write the function \verb^strlen()^
\lstinputlisting[style=basicc]{../Code/ChapI/strlen.c}
\end{column}

\end{columns}
\end{frame}


\endinput

%%%%%%%%%%%%%%%%%%%%%%%%%%%%%%%%%%%%%%%%%%%%%%%%%%%%%%%%%%%%%%%%%%


%\begin{center}
%\end{center}
%\vspace{-0.5in}
%\section*{Strings}
%
%\end{itemize}

%\newpage
%{\samepage
%\begin{itemize}
%\item \begin{verbatim}
%%char w[250] = "abc";
%\end{verbatim}
%\item \begin{verbatim}
char *w = "abc";
%\end{verbatim}
%\end{itemize}
%\begin{verbatim}
%
%#include <stdio.h>
%#include <ctype.h>
%
%int main(void)
%{
   %char w1[100] = "test";
   %char *w2 = "test";
%
   %printf("%s -> ", w1);
   %w1[0] = toupper(w1[0]);
   %printf("%s\n", w1);
%
   %printf("%s -> ", w2);
   %/* Seg Faults */
   %w2[0] = toupper(w2[0]);
   %printf("%s\n", w2);
%
   %return 0;
%}
%\end{verbatim}
%}
%
