\section{F: Data Types, Maths and Characters}

%------------------------------------------------

\begin{frame}[fragile]
\frametitle{Fundamental Data types}

\begin{columns}

\begin{column}[c]{0.35\textwidth}
{\small
\begin{itemize}[<+->]
\item {\bf [} unsigned $|$ signed {\bf ]}
\item {\bf [} long $|$ short {\bf ]}
\item {\bf [} char | int $|$ float $|$ double {\bf ]}
\item The use of {\tt int} implies {\tt signed int} without the need to state it.
\item Likewise {\tt unsigned short} means {\tt unsigned short int}.
\end{itemize}
}
\end{column}

\begin{column}[c]{0.60\textwidth}
{\tiny
\begin{tabular}{|l|l|l|} \hline
Type & Minimum size (bits) & Format specifier \\ \hline
%
char & 8 & \%c \\ \hline
%
signed char & 8 & \%c (or \%hhi for numerical output) \\ \hline
%
unsigned char & 8 & \%c (or \%hhu for numerical output) \\ \hline
%
short & 16 & \%hi or \%hi \\
short int &  &  \\
signed short &  &  \\
signed short int &  &  \\ \hline
%
unsigned short & 16 & \%hu \\
unsigned short int &  &  \\ \hline
int & 16 & \%i or \%d \\
signed &  &  \\
signed int &  &  \\ \hline
%
unsigned & 16 & \%u \\
unsigned int &  &  \\ \hline
%
long & 32 & \%ld or \%li \\
long int &  &  \\
signed long &  &  \\
signed long int &  &  \\ \hline
%
unsigned long & 32 & \%lu \\
unsigned long int &  &  \\ \hline
long long & 64 & \%lli or \%lld \\
long long int &  &  \\
signed long long &  &  \\
signed long long int &  &  \\ \hline
%
unsigned long long & 64 & \%llu \\
unsigned long long int &  &  \\ \hline
%
float &  & scanf(): \\
 &  & \%f, \%g, \%e, \%a \\ \hline
%
double &  & \%lf, \%lg, \%le, \%la \\ \hline 
%
long double & & \%Lf, \%Lg, \%Le, \%La \\ \hline
\end{tabular}
}
\end{column}

\end{columns}
\end{frame}

%------------------------------------------------

\begin{frame}[fragile]
\frametitle{Binary Storage of Numbers}

\begin{columns}

\begin{column}{0.45\textwidth}
In an unsigned char :
\begin{center}
\begin{tabular}{|c|c|c|c|c|c|c|c|}\hline
$2^7$ & $2^6$ & $2^5$ & $2^4$ & $2^3$ & $2^2$ & $2^1$ & $2^0$ \\ \hline
0     & 1     & 0     & 0     & 1     & 1     & 0     & 0     \\ \hline
\end{tabular}
\end{center}

The above represents : $1 * 64 + 1 * 8 + 1 * 4 = 76$.

\begin{itemize}[<+->]
\item Floating operations need not be exact.
\end{itemize}
\lstinputlisting[style=basicc]{../Code/ChapF/prec1.c}
\outputlisting{../Code/ChapF/prec1.autoout}
\end{column}

\pause
\begin{column}{0.45\textwidth}
\begin{itemize}[<+->]
\item Not all floats are representable so are only approximated.
\item Since floats may not be stored exactly, it doesn't make sense to try and compare them:
\begin{lstlisting}[style=basicc,numbers=none]
if ( d == 0.3 )
\end{lstlisting}
\item Therefore, we don't allow this by explicitly using the compiler warning flag: {\tt -Wfloat-equal}
\end{itemize}
\end{column}

\end{columns}
\end{frame}

%------------------------------------------------

\begin{frame}[fragile]
\frametitle{\tt sizeof()}

\begin{columns}

\begin{column}{0.45\textwidth}
To find the exact size in bytes of a type on a particular machine,
use {\tt sizeof()}. On a Dell Windows 10 laptop running WSL:
\lstinputlisting[style=basicc]{../Code/ChapF/sizeof.c}
\end{column}

\pause
\begin{column}{0.40\textwidth}
\outputlisting{../Code/ChapF/sizeof.autoout}
\end{column}

\end{columns}
\end{frame} 

%------------------------------------------------

\begin{frame}[fragile]
\frametitle{Mathematical Functions}

\begin{itemize}[<+->]
\item There are no mathematical functions built into the C language.
\item However, there are many functions in the maths library
which may linked in using the {\bf -lm} option with the compiler.
\item Functions include :
\begin{verbatim}
sqrt()  pow() round() 
fabs() exp()  log()
sin()  cos()  tan()
\end{verbatim}
\item Most take {\tt double}s as arguments and return {\tt double}s.
\end{itemize}

\end{frame}

%------------------------------------------------

\begin{frame}[fragile]
\frametitle{Casting}
\begin{columns}

\begin{column}{0.45\textwidth}
\lstinputlisting[style=basicc]{../Code/ChapF/math1.c}
\outputlisting{../Code/ChapF/math1.manout}
\end{column}

\begin{column}{0.45\textwidth}
\begin{itemize}[<+->]
\item An explicit type conversion is called a {\it cast}.
\item {\it If it moves - cast it}. Don't trust the compiler to do it for you~!
\end{itemize}
\end{column}


\end{columns}
\end{frame}

%------------------------------------------------

\begin{frame}[fragile]
\frametitle{Storage of Characters}
\begin{columns}

\begin{column}{0.45\textwidth}
\begin{itemize}[<+->]
\item Characters are stored in the machine as one byte (generally $8$-bits storing one
of {\bf 256} possible values).
\item These may be thought of a characters, or very small integers.
\item Only a subset of these 256 values are required
for the printable characters, space, newline etc.
\item Declaration:
\begin{verbatim}
char c;

c = 'A';
\end{verbatim}
or :
\begin{verbatim}
char c1 = 'A', c2 = '*', c3 = ';' ;
\end{verbatim}
\end{itemize}
\end{column}

\pause
\begin{column}{0.45\textwidth}
\begin{itemize}[<+->]
\item The particular integer used to represent a character
is dependent on the encoding used. The most common
of these, used on most UNIX and PC platforms, is ASCII.

\begin{center}
\begin{tabular}{|l|lllll|}\hline
lowercase   & 'a' & 'b' & 'c' & \verb^...^ & 'z' \\
ASCII value & 97  &  98 &  99 & \verb^...^ & 112 \\ \hline
uppercase   & 'A' & 'B' & 'C' & \verb^...^ & 'Z' \\
ASCII value & 65  &  66 &  67 & \verb^...^ &  90 \\ \hline
digit       & '0' & '1' & '2' & \verb^...^ & '9' \\
ASCII value & 48  &  49 &  50 & \verb^...^ &  57 \\ \hline
other       & '\&' & '*' & '+' & \verb^...^ & \\
ASCII value & 38  &  42 &  43 & \verb^...^ & \\ \hline
\end{tabular}
\end{center}
\end{itemize}
\end{column}

\end{columns}
\end{frame}


%%%%%%%%%%%%%%%%%%%%%%%%%%%%%%%%%%%%%%%%%%%%%%%%%%%%%%%%%%%%%%%%%%


\begin{frame}[fragile]
\frametitle{Using Characters}
\begin{columns}

\begin{column}{0.45\textwidth}
\begin{itemize}[<+->]
\item When using \verb^printf()^ and \verb^scanf()^ the formats
\verb^%c^ and \verb^%i^ do very different things :
\begin{verbatim}
char c = 'a'
printf("%c\n", c); /* prints : a  */
printf("%i\n", c); /* prints : 97 */
\end{verbatim}
\item Hard-to-print characters have an escape sequence
i.e. to print a newline, the 2 character escape \verb^'\n'^ is used.
\end{itemize}
\end{column}

\begin{column}{0.55\textwidth}
{\small
\begin{tabular}{|l|l|l|}\hline
Escape sequence	& Hex value & Character \\ \hline
\textbackslash a & 07 & Alert (Beep, Bell)\\
\textbackslash b &08&Backspace\\
\textbackslash e &1B&Escape character\\
\textbackslash f &0C&Formfeed Page Break\\
\textbackslash n &0A&Newline (Line Feed)\\
\textbackslash r &0D&Carriage Return\\
\textbackslash t &09&Horizontal Tab\\
\textbackslash v &0B&Vertical Tab\\
\textbackslash\textbackslash &5C&Backslash\\
\textbackslash ' &27&Apostrophe \\
\textbackslash " &22&Double quote \\
\textbackslash ? &3F&Question mark \\ \hline
\end{tabular}
}
\end{column}

\end{columns}
\end{frame}


%%%%%%%%%%%%%%%%%%%%%%%%%%%%%%%%%%%%%%%%%%%%%%%%%%%%%%%%%%%%%%%%%%


\begin{frame}[fragile]
\frametitle{Using {\tt getchar()} and {\tt putchar()}}
\begin{columns}[T]
\begin{column}{0.45\textwidth}

\lstinputlisting[style=basicc]{../Code/ChapF/dblout1.c}
\outputlisting{../Code/ChapF/dblout1.manout}

{\footnotesize
This has the unfortunate problem of requiring a `special' character to terminate.
More aggressively, the user could terminate by pressing {\tt CTRL-C}.
}
\end{column}

\pause
\begin{column}{0.45\textwidth}
\lstinputlisting[style=basicc]{../Code/ChapF/dblout2.c}
\outputlisting{../Code/ChapF/dblout2.manout}

{\footnotesize
The end-of-file constant is defined in {\tt stdio.h}. Although
system dependent, {\tt -1} is often used. On the UNIX system
this is generated when the end of a file being piped is reached,
or when {\tt CTRL-D} is pressed.
}
\end{column}
\end{columns}
\end{frame}


%%%%%%%%%%%%%%%%%%%%%%%%%%%%%%%%%%%%%%%%%%%%%%%%%%%%%%%%%%%%%%%%%%

\begin{frame}[fragile]
\frametitle{Capitalization}
\begin{columns}[T]
\begin{column}{0.45\textwidth}

\lstinputlisting[style=basicc]{../Code/ChapF/caps1.c}
\outputlisting{../Code/ChapF/caps1.manout}

{\footnotesize
This is more easily achieved by using some of
the definitions found in {\tt ctype.h}.
}
\end{column}


\pause
\begin{column}{0.45\textwidth}
\begin{center}
\begin{tabular}{|l|l|} \hline
Macro        & {\tt true} returned if: \\ \hline
isalnum(int c) & Letter or digit \\
isalpha(int c) & Letter \\
iscntrl(int c) & Control character \\
isdigit(int c) & Digit \\
isgraph(int c) & Printable (not space) \\
islower(int c) & Lowercase\\
isprint(int c) & Printable\\
ispunct(int c) & Punctuation\\
isspace(int c) & White Space\\
isupper(int c) & Uppercase\\
isxdigit(int c)& Hexadecimal\\
isascii(int c) & ASCII code \\ \hline
\end{tabular}
\end{center}
\end{column}

\end{columns}
\end{frame}

%%%%%%%%%%%%%%%%%%%%%%%%%%%%%%%%%%%%%%%%%%%%%%%%%%%%%%%%%%%%%%%%%%


\begin{frame}[fragile]
\frametitle{{\tt ctype.h}}
\begin{columns}[T]

\begin{column}{0.45\textwidth}
Some useful functions are :
\begin{center}
\begin{tabular}{|l|l|} \hline
Function/Macro & Returns: \\ \hline
int tolower(int c) & Lowercase c \\
int toupper(int c) & Uppercase c \\
int toascii(int c) & ASCII code for c \\ \hline
\end{tabular}
\end{center}
\lstinputlisting[style=basicc]{../Code/ChapF/caps2.c}
\end{column}

\pause
\begin{column}{0.45\textwidth}
\lstinputlisting[style=basicc]{../Code/ChapF/caps3.c}
\outputlisting{../Code/ChapF/caps3.manout}
\end{column}

\end{columns}
\end{frame}
