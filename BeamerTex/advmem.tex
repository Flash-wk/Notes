\section{L : Advanced Memory Handling}
\label{chap:advmem}

%%%%%%%%%%%%%%%%%%%%%%%%%%%%%%%%%%%%%%%%%%%%%%%%%%%%%%%%%%%%%%

\begin{frame}[fragile]
\frametitle{String Constants}
\begin{columns}[T]

\begin{column}{0.35\textwidth}
\lstinputlisting[style=basicc]{../Code/ChapL/nify1.c}
\end{column}

\pause
\begin{column}{0.55\textwidth}
\begin{itemize}[<+->]
\item This looks (at first) like a sensible attempt to accept a string and change it {\em in-place} to capitalise all `n' characters.
It crashes though via a segmentation fault.
\item With the usual compile flags we get no more information.
\item But using:
\begin{verbatim}
gcc nify1.c -g3 -fsanitize=undefined
-fsanitize=address -o nify1
\end{verbatim}
we find that
\lstinputlisting[style=basicc,numbers=none,linerange={23-23}]{../Code/ChapL/nify1.c}
is the culprit.
\item It turns out that in \verb^main()^ we have passed a {\bf constant} string to the function. This is in a part of memory that we have read-only permission.
\end{itemize}
\end{column}

\end{columns}
\end{frame}

%%%%%%%%%%%%%%%%%%%%%%%%%%%%%%%%%%%%%%%%%%%%%%%%%%%%%%%%%%%%%%

\begin{frame}[fragile]
\frametitle{Local Variables}
\begin{columns}[T]

\begin{column}{0.35\textwidth}
\lstinputlisting[style=basicc]{../Code/ChapL/nify2.c}
\end{column}

\pause
\begin{column}{0.55\textwidth}
\begin{itemize}[<+->]
\item Now we try to create a copy of the string, and return a pointer to it.
\item With the usual compile flags we're told:
\begin{verbatim}
nify2.c: In function `nify':
nify2.c:33:11: warning: function returns address
of local variable [-Wreturn-local-addr]
   33 |    return t;
\end{verbatim}
\item The string \verb^t^ is local to \verb^nify()^.
\item What happens in this memory when outside the scope of this function is completely undefined.
\end{itemize}
\end{column}

\end{columns}
\end{frame}

%%%%%%%%%%%%%%%%%%%%%%%%%%%%%%%%%%%%%%%%%%%%%%%%%%%%%%%%%%%%%%

\begin{frame}[fragile]
\frametitle{Static Variables}
\begin{columns}[T]

\begin{column}{0.35\textwidth}
\lstinputlisting[style=basicc]{../Code/ChapL/nify3.c}
\end{column}

\pause
\begin{column}{0.55\textwidth}
\begin{itemize}[<+->]
\item We could just make the local string a \verb^static^ and return it's address couldn't we?
\item This only works if we're very careful with the order in which we use the strings.
\item This code fails because, in \verb^main()^, by the time we \verb^strcmp(s1, "iNcoNveNieNciNg")^ the contents of \verb^s1^ have been overwritten by \verb^"Neill"^. 
\item The pointers \verb^s1^ and \verb^s2^ are the same.
\end{itemize}
\end{column}

\end{columns}
\end{frame}

%%%%%%%%%%%%%%%%%%%%%%%%%%%%%%%%%%%%%%%%%%%%%%%%%%%%%%%%%%%%%%

\begin{frame}[fragile]
\frametitle{Using {\em malloc()}}
\begin{columns}[T]

\begin{column}{0.55\textwidth}
\begin{itemize}[<+->]
\item We must use \verb^malloc()^ instead.
\item
\begin{verbatim}
void* malloc(int n);
\end{verbatim}
allocates $n$ bytes and returns a pointer to the allocated
memory. The memory is  not initialized.
\item Now, when our function is called, a dedicated chunk of memory is allocated.
\item This memory is always in scope until \verb^free()^ is used on it.
\item We must free the memory somewhere though, otherwise memory leaks develop.
\item We will see \verb^calloc()^ (and perhaps \verb^realloc()^) later.
\end{itemize}
\end{column}

\pause
\begin{column}{0.35\textwidth}
\lstinputlisting[style=basicc]{../Code/ChapL/nify4.c}
\end{column}

\end{columns}
\end{frame}

%%%%%%%%%%%%%%%%%%%%%%%%%%%%%%%%%%%%%%%%%%%%%%%%%%%%%%%%%%%%%%

\begin{frame}[fragile]
\frametitle{Variable Length Arrays}
\begin{columns}[T]

\begin{column}{0.35\textwidth}
\lstinputlisting[style=basicc]{../Code/ChapL/strdup1.c}
\end{column}

\pause
\begin{column}{0.55\textwidth}
\begin{itemize}[<+->]
\item Here we duplicate a string into $t$.
\item This is known as a variable length array.
\item However, we will always use the \verb^-Wvla^ with the compiler to prevent them.
\item There are a number of reasons for this:
\begin{itemize}[<+->]
\item Some C++ compilers don't accept it.
\item The memory comes off the stack not the heap, and
you have no idea if the allocation has worked (it'll just crash if not)
\item \url{https://nullprogram.com/blog/2019/10/27/}
\end{itemize}
\item None of these is a problem if we use \verb^malloc()^.
\end{itemize}
\end{column}

\end{columns}
\end{frame}


%%%%%%%%%%%%%%%%%%%%%%%%%%%%%%%%%%%%%%%%%%%%%%%%%%%%%%%%%%%%%%

\begin{frame}[fragile]
\frametitle{Memory Leaks}
\begin{columns}[T]

\begin{column}{0.45\textwidth}
\lstinputlisting[style=basicc]{../Code/ChapL/strdup2.c}
\end{column}

\pause
\begin{column}{0.45\textwidth}
\begin{itemize}[<+->]
\item This code appears to work correctly.
\item However, it actually {\bf leaks}. The memory
allocated was never \verb^free()^'d.
\item This is best found by running the program \verb^valgrind^.
{\tiny
\begin{verbatim}
String String
==474==
==474== HEAP SUMMARY:
==474==     in use at exit: 7 bytes in 1 blocks
==474==   total heap usage: 2 allocs, 1 frees, 1,031 bytes allocated
==474==
==474== LEAK SUMMARY:
==474==    definitely lost: 7 bytes in 1 blocks
\end{verbatim}
}
\end{itemize}
\end{column}

\end{columns}
\end{frame}

%%%%%%%%%%%%%%%%%%%%%%%%%%%%%%%%%%%%%%%%%%%%%%%%%%%%%%%%%%%%%%

\begin{frame}[fragile]
\frametitle{{\em free()}}
\begin{columns}[T]

\begin{column}{0.45\textwidth}
\lstinputlisting[style=basicc]{../Code/ChapL/strdup3.c}
\end{column}

\pause
\begin{column}{0.45\textwidth}
\begin{itemize}[<+->]
\item This code is now correct.
{\tiny
\begin{verbatim}
String String
==475==
==475== HEAP SUMMARY:
==475==     in use at exit: 0 bytes in 0 blocks
==475==   total heap usage: 2 allocs, 2 frees, 1,031 bytes allocated
==475==
==475== All heap blocks were freed -- no leaks are possible
\end{verbatim}
}
\end{itemize}
\end{column}

\end{columns}
\end{frame}


%%%%%%%%%%%%%%%%%%%%%%%%%%%%%%%%%%%%%%%%%%%%%%%%%%%%%%%%%%%%%%

\begin{frame}[fragile]
\frametitle{Structures with Self-Referential Pointers}
\begin{columns}[T]

\begin{column}{0.45\textwidth}
\lstinputlisting[style=basicc]{../Code/ChapL/llist1.c}
\end{column}

\pause
\begin{column}{0.45\textwidth}
\begin{itemize}[<+->]
\item The structure contains a pointer to a something of it's own type (even before we've fully defined the struture itself).
\item Here, if \verb^p^ points to \verb^a^, then \verb^p->next->next^ points to \verb^c^.
\end{itemize}
\end{column}

\end{columns}
\end{frame}


%%%%%%%%%%%%%%%%%%%%%%%%%%%%%%%%%%%%%%%%%%%%%%%%%%%%%%%%%%%%%%

\begin{frame}[fragile]
\frametitle{Linked Lists}
\begin{columns}[T]

\begin{column}{0.45\textwidth}
\lstinputlisting[style=basicc,numbers=none,linerange={1-32}]{../Code/ChapL/llist2.c}
\end{column}

\pause
\begin{column}{0.45\textwidth}
\lstinputlisting[style=basicc,numbers=none,linerange={34-50}]{../Code/ChapL/llist2.c}
\outputlisting{../Code/ChapL/llist2.autoout}
\begin{itemize}[<+->]
\item \verb^calloc()^ is similar to \verb^malloc()^, but {\em clears} the memory is reserves for you. It's passed the number of array cells you wish to create, and the size of each of them.
\end{itemize}
\end{column}

\end{columns}
\end{frame}


%%%%%%%%%%%%%%%%%%%%%%%%%%%%%%%%%%%%%%%%%%%%%%%%%%%%%%%%%%%%%%
