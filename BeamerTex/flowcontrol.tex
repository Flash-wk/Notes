\section{D: Flow Control}

\begin{frame}[fragile]
\frametitle{Comparisons}
\begin{columns}

\begin{column}{0.5\textwidth}
\begin{center}
\begin{tabular}{|l|l|} \hline
\verb^<^    & less than             \\ \hline
\verb^>^    & greater than          \\ \hline
\verb^<=^   & less than or equal to     \\ \hline
\verb^>=^   & greater than or equal to  \\ \hline
\verb^==^   & {\bf equal to}        \\ \hline
\verb^!=^   & not equal to          \\ \hline
\verb^!^    & not               \\ \hline
\verb^&&^   & logical AND           \\ \hline
\verb^||^   & logical OR            \\ \hline
\end{tabular}
\end{center}
\end{column}

\begin{column}{0.5\textwidth}
\begin{itemize}[<+->]
\item Any relation is either {\it true} or {\it false}.
\item Any non-zero value is {\it true}.
\item \verb^(a < b)^ returns the value {\it 0} or {\it 1}.
\item \verb^(i == 5)^ is a {\bf test} not an {\bf assignment}.
\item \verb^(!a)^ is either {\it true (1)} or {\it false (0)}.
\item \verb^(a && b)^ is {\it true} if both \verb^a^ and \verb^b^ are {\it true}.
\item Single \verb^&^ and \verb^|^ are {\it bitwise} operators not comparisons - more on this later.
\end{itemize}
\end{column}

\end{columns}
\end{frame}

\begin{frame}[fragile]
\frametitle{Short-Circuit Evaluation}

\begin{lstlisting}[style=basicc,numbers=none]
if(x >= 0.0 && sqrt(x) < 10.0){

..... /* Do Something */

}
\end{lstlisting}

It's not possible to take the \verb^sqrt()^ of a negative number.
Here, the \verb^sqrt()^ statement is never reached if the first test is {\it false}.
In a logical AND, once any expression is {\it false}, the whole must
be {\it false}.
\end{frame}


\begin{frame}[fragile]
\frametitle{The {\tt if()} Statement}

\begin{columns}

\begin{column}{0.45\textwidth}
Strictly, you don't need braces if there is only one statement as part of the {\tt if}~:
\begin{lstlisting}[style=basicc,numbers=none]
if (expr)
   statement
\end{lstlisting}

If more than one statement is required~:
\begin{lstlisting}[style=basicc,numbers=none]
if (expr) {
   statement-1
       .
       .
       .
   statement-n
}
\end{lstlisting}
However, we will {\bf always} brace them, even if it's not necessary.
\end{column}

\begin{column}{0.45\textwidth}
Adding an \verb^else^ statement~:

\begin{lstlisting}[style=basicc,numbers=none]
if (expr) {
   statement-1
       .
       .
       .
   statement-n
}
else{
   statement-a
       .
       .
       .
   statement-e
}
\end{lstlisting}
\end{column}

\end{columns}
\end{frame}

\begin{frame}[fragile]
\frametitle{A Practical Example of {\tt if}:}
\begin{columns}

\begin{column}{0.5\textwidth}
\lstinputlisting[style=basicc]{../Code/ChapD/min.c}
\end{column}

\begin{column}{0.40\textwidth}
\outputlisting{../Code/ChapD/min.manout}
\end{column}

\end{columns}
\end{frame}

\begin{frame}[fragile]
\frametitle{The {\tt while()} Statement}
\begin{columns}

\begin{column}{0.45\textwidth}
\begin{lstlisting}[style=basicc,numbers=none]
while(expr)
   statement
\end{lstlisting}

This, as with the {\tt for} loop, may execute compound statements~:

\begin{lstlisting}[style=basicc,numbers=none]
while(expr){
   statement-1
       .
       .
       .
   statement-n
}
\end{lstlisting}
However, we will {\bf always} brace them, even if it's not necessary.
\end{column}

\begin{column}{0.45\textwidth}
\lstinputlisting[style=basicc]{../Code/ChapD/while.c}
\outputlisting{../Code/ChapD/while.autoout}
\end{column}

\end{columns}
\end{frame}

\begin{frame}[fragile]
\frametitle{The {\tt for()} Loop}
\begin{columns}

\begin{column}{0.45\textwidth}
This is one of the more complex and heavily used means
for controlling execution flow.

\begin{lstlisting}[style=basicc,numbers=none]
for( init ; test; loop){
   statement-1
       .
       .
       .
   statement-n
}
\end{lstlisting}

and may be thought of as~:
\begin{lstlisting}[style=basicc,numbers=none]
init;
while(test){
   statement-1
       .
       .
       .
   statement-n
   loop;
}
\end{lstlisting}
\end{column}

\begin{column}{0.45\textwidth}
In the \verb^for()^ loop, note~:
\begin{itemize}[<+->]
\item Semi-colons separate the three parts.
\item Any (or all) of the three parts could be empty.
\item If the test part is empty, it evaluates to {\it true}.
\item \verb^for(;;){ a+=1; }^  is an infinite loop.
\end{itemize}
\end{column}

\end{columns}
\end{frame}

\begin{frame}[fragile]
\frametitle{A Triply-Nested Loop}

\begin{columns}
\begin{column}{0.55\textwidth}
\lstinputlisting[style=basicc]{../Code/ChapD/triple.c}
\end{column}

\begin{column}{0.40\textwidth}
{\scriptsize Output~:}
\lstinputlisting[style=basicterm,linerange={1-8}]{../Code/ChapD/triple.autoout}
{\scriptsize etc.}
\lstinputlisting[style=basicterm,linerange={30-38}]{../Code/ChapD/triple.autoout}
\end{column}

\end{columns}
\end{frame}

\begin{frame}[fragile]
\frametitle{The Comma Operator}

This has the lowest precedence of all the operators in C
and associates left-to-right.

\begin{lstlisting}[style=basicc,numbers=none]
a = 0 , b = 1;
\end{lstlisting}

Hence, the {\tt for} loop may become quite complex~:

\begin{lstlisting}[style=basicc,numbers=none]
for(sum = 0, i = 1; i <= n; ++i){
   sum += i;
}
\end{lstlisting}

An equivalent, but more difficult to read expression~:
\begin{lstlisting}[style=basicc,numbers=none]
for(sum = 0 , i = 1; i <= n; ++i, sum += i);
\end{lstlisting}

Notice the loop has an empty body, hence the semicolon.
\end{frame}

\begin{frame}[fragile]
\frametitle{The {\tt do-while()} Loop}

\begin{columns}
\begin{column}{0.35\textwidth}
\begin{lstlisting}[style=basicc,numbers=none]
do {
   statement-1
       .
       .
       .
   statement-n
} while ( test );
\end{lstlisting}

Unlike the \verb^while()^ loop, the \verb^do-while()^ will always
be executed at least once.
\end{column}

\begin{column}{0.60\textwidth}
\lstinputlisting[style=basicc]{../Code/ChapD/dowhile.c}
\outputlisting{../Code/ChapD/dowhile.autoout}
\end{column}

\end{columns}
\end{frame}

\begin{frame}[fragile]
\frametitle{The {\tt switch()} Statement}
\begin{columns}

\begin{column}{0.45\textwidth}
\begin{lstlisting}[style=basicc,numbers=none]
switch (val) {
   case 1 :
      a++;
      break;
   case 2 :
   case 3 :
      b++;
      break;
   default :
      c++;
}
\end{lstlisting}
\end{column}

\begin{column}{0.45\textwidth}
\begin{itemize}[<+->]
\item The \verb^val^ must be an integer.
\item The \verb^break^ statement causes execution to jump out
of the loop. No \verb^break^ statement causes execution to
`fall through' to the next line.
\item The \verb^default^ label is a catch-all.
\end{itemize}
\end{column}

\end{columns}
\end{frame}

\begin{frame}[fragile]
\frametitle{The {\tt switch()} Statement}
\begin{columns}

\begin{column}{0.55\textwidth}
\lstinputlisting[style=basicc]{../Code/ChapD/switch.c}
\end{column}

\begin{column}{0.40\textwidth}
\outputlisting{../Code/ChapD/switch.manout}
\end{column}

\end{columns}
\end{frame}

\begin{frame}[fragile]
\frametitle{The Conditional {\tt (?)} Operator}
\begin{columns}

\begin{column}{0.40\textwidth}
As we have seen, C programers have a range of techniques available to reduce the amount of typing~:
\begin{lstlisting}[style=basicc,numbers=none]
expr1 ? expr2 : expr3
\end{lstlisting}

If {\tt expr1} is {\it true} then {\tt expr2} is executed, else
{\tt expr3} is evaluated.
\end{column}

\begin{column}{0.55\textwidth}
\lstinputlisting[style=basicc]{../Code/ChapD/minq.c}
\end{column}

\end{columns}
\end{frame}
