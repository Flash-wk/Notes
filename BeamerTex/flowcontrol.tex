\section*{D: Flow Control}

\begin{frame}[fragile]
\frametitle{Comparisons}
\begin{columns}

\begin{column}{0.5\textwidth}
\begin{center}
\begin{tabular}{|l|l|} \hline
\verb^<^    & less than             \\ \hline
\verb^>^    & greater than          \\ \hline
\verb^<=^   & less than or equal to     \\ \hline
\verb^>=^   & greater than or equal to  \\ \hline
\verb^==^   & {\bf equal to}        \\ \hline
\verb^!=^   & not equal to          \\ \hline
\verb^!^    & not               \\ \hline
\verb^&&^   & logical AND           \\ \hline
\verb^||^   & logical OR            \\ \hline
\end{tabular}
\end{center}
\end{column}

\begin{column}{0.5\textwidth}
\begin{itemize}[<+->]
\item Any relation is either {\it true} or {\it false}.
\item Any non-zero value is {\it true}.
\item \verb^(a < b)^ returns the value {\it 0} or {\it 1}.
\item \verb^(i == 5)^ is a {\bf test} not an {\bf assignment}.
\item \verb^(!a)^ is either {\it true (1)} or {\it false (0)}.
\item \verb^(a && b)^ is {\it true} if both \verb^a^ and \verb^b^ are {\it true}.
\item Single \verb^&^ and \verb^|^ are {\it bitwise} operators not comparisons - more on this later.
\end{itemize}
\end{column}

\end{columns}
\end{frame}

\begin{frame}[fragile]
\frametitle{Short-Circuit Evaluation}

\begin{lstlisting}[style=basicc,numbers=none]
if(x >= 0.0 && sqrt(x) < 10.0){

..... /* Do Something */

}
\end{lstlisting}

It's not possible to take the \verb^sqrt()^ of a negative number.
Here, the \verb^sqrt()^ statement is never reached if the first test is {\it false}.
In a logical AND, once any expression is {\it false}, the whole must
be {\it false}.
\end{frame}


\begin{frame}[fragile]
\frametitle{The {\tt if()} Statement}

\begin{verbatim}
if (expr)
   statement
\end{verbatim}

If more than one statement is required:

\begin{verbatim}
if (expr) {
   statement-1
       .
       .
       .
   statement-n
}
\end{verbatim}

Adding an \verb^else^ statement:

\begin{verbatim}
if (expr) {
   statement-1
       .
       .
       .
   statement-n
}
else{
   statement-a
       .
       .
       .
   statement-e
}
\end{verbatim}

A practical example:

\begin{verbatim}
#include <stdio.h>

int main(void)
{
   int   x, y, z, min;

   printf("Input three integers:  ");
   scanf("%d%d%d", &x, &y, &z);
   if (x < y)
      min = x;
   else
      min = y;
   if (z < min)
      min = z;
   printf("The minimum value is " \
          "%d\n", min);
   return 0;
}
\end{verbatim}
\end{frame}

\begin{frame}[fragile]
\frametitle{The {\tt while()} Statement}

\begin{verbatim}
while(expr)
   statement
\end{verbatim}

This, as with the \verb^for^ loop, may execute compound statements:

\begin{verbatim}
while(expr){
   statement-1
       .
       .
       .
   statement-n
}
\end{verbatim}
\end{frame}

\begin{frame}[fragile]
\frametitle{The {\tt for()} Loop}

This is one of the more complex and heavily used means
for controlling execution flow.

\begin{verbatim}
for( init ; test; loop){
   statement-1
       .
       .
       .
   statement-n
}
\end{verbatim}

and may be thought of as:
\begin{verbatim}
init;
while(test){
   statement-1
       .
       .
       .
   statement-n
   loop;
}
\end{verbatim}

In the \verb^for()^ loop, note:
\begin{itemize}[<+->]
\item Semi-colons separate the three parts.
\item Any (or all) of the three parts could be empty.
\item If the test part is empty, it evaluates to {\it true}.\\
\verb^for(;;){ a+=1; }^  is an infinite loop.
\end{itemize}

{\samepage
\begin{verbatim}
/* Find triples of integers that
   add up to N.                  */

#include <stdio.h>

#define   N   7

int main(void)
{
   int   cnt = 0, i, j, k;

   for (i = 0; i <= N; i++)
      for (j = 0; j <= N; j++)
       for (k = 0; k <= N; k++)
          if (i + j + k == N) {
             ++cnt;
             printf("%3d%3d%3d\n",
                    i, j, k);
          }
   printf("\nCount: %d\n", cnt);
   return 0;
}
\end{verbatim}
}
The result is:

{\tiny
\begin{verbatim}
  0  0  7
  0  1  6
  0  2  5
  0  3  4
  0  4  3
  0  5  2
  0  6  1
  0  7  0
  1  0  6
  1  1  5
  1  2  4
  1  3  3
  1  4  2
  1  5  1
  1  6  0
  2  0  5
  2  1  4
  2  2  3
  2  3  2
  2  4  1
  2  5  0
  3  0  4
  3  1  3
  3  2  2
  3  3  1
  3  4  0
  4  0  3
  4  1  2
  4  2  1
  4  3  0
  5  0  2
  5  1  1
  5  2  0
  6  0  1
  6  1  0
  7  0  0
\end{verbatim}
}
\end{frame}

\begin{frame}[fragile]
\frametitle{The Comma Operator}

This has the lowest predence of all the operators in C
and associates left to right.

\begin{verbatim}
a = 0 , b = 1;
\end{verbatim}

Hence, the \verb^for^ loop may become quite complex :

\begin{verbatim}
for(sum = 0 , i = 1; i <= n; ++i)
   sum += i;
\end{verbatim}

An equivalent, but more difficult to read expression :

{\small
\begin{verbatim}
for(sum = 0 , i = 1; i <= n; ++i, sum += i);
\end{verbatim}
}

\begin{itemize}[<+->]
\item Notice the loop has an empty body, hence the semicolon.
\end{itemize}
\end{frame}

\begin{frame}[fragile]
\frametitle{The {\tt do-while()} Statement}

\begin{verbatim}
do {
   statement-1
       .
       .
       .
   statement-n
} while ( test );
\end{verbatim}

Unlike the \verb^while()^ loop, the \verb^do-while()^ will always
be executed at least once.
\end{frame}

\begin{frame}[fragile]
\frametitle{The {\tt switch()} Statement}

\begin{verbatim}
switch (val) {
   case 1 :
      a++;
      break;
   case 2 :
   case 3 :
      b++;
      break;
   default :
      c++;
}
\end{verbatim}

\begin{itemize}[<+->]
\item The \verb^val^ must be an integer.
\item The \verb^break^ statement causes execution to jump out
of the loop. No \verb^break^ statement causes execution to
`fall through' to the next line.
\item The \verb^default^ label is a catch-all.
\end{itemize}
\end{frame}

\begin{frame}[fragile]
\frametitle{The Conditional Operator}

\begin{verbatim}
expr1 ? expr2 : expr3
\end{verbatim}

If \verb^expr1^ is {\it true} then \verb^expr2^ is executed, else
\verb^expr3^ is evaluated, i.e.:

\verb^x = ((y < z) ? y : z);^
\end{frame}
