\section{C: Grammar}

\begin{frame}[fragile]
\frametitle{Grammar}
\begin{itemize}[<+->]
\item  C has a grammar/syntax like every other language.
\item  It has {\it Keywords}, {\it Identifiers}, {\it Constants}, {\it String Constants}, {\it Operators} and {\it Punctuators}.
\item
Valid Identifiers~:\\
\verb+k+,
\verb+_id+,
\verb+iamanidentifier2+,
\verb+so_am_i+.
\item  {\bf Invalid} Identifiers~:\\
\verb+not#me+,
\verb+101_south+,
\verb+-plus+.
\item  Constants~:\\
\verb+17+ (decimal),
\verb+017+ (octal),
\verb+0x17+ (hexadecimal).
\item  String Constant enclosed in double-quotes~:\\
\verb+"I am a string"+
\end{itemize}
\end{frame}

\begin{frame}[fragile]
\frametitle{Operators}

\begin{columns}
\begin{column}{0.5\textwidth}
\begin{itemize}[<+->]
\item All operators have rules of both {\it precedence}
and {\it associativity}.
\item  \verb$1 + 2 * 3$ is the same as
\verb$1 + (2 * 3)$ because \verb$*$ has
a higher precedence than \verb$+$.
\item  The associativity of \verb$+$ is left-to-right, thus\\
\verb$1 + 2 + 3$ is equivalent to
\verb$(1 + 2) + 3$.
\item  Increment and decrement operators~:\\
\verb$i++;$ is equivalent to \verb$i = i + 1;$
\item  May also be prefixed \verb$--i;$
\end{itemize}
\end{column}

\pause
\begin{column}{0.5\textwidth}
\lstinputlisting[style=basicc]{../Code/ChapC/prepostfix.c}
{Question : What is the output~?}
\end{column}

\end{columns}
\end{frame}

\begin{frame}[fragile]
\frametitle{Assignment}

\begin{columns}
\begin{column}{0.5\textwidth}
\begin{itemize}[<+->]
\item The \verb$=$ operator has a low precedence
and a right-to-left associativity.
\item  \verb$a = b = c = 0;$ is valid and equivalent to~:\\
\verb$a = (b = (c = 0));$
\item  \verb$i = i + 3;$ is the same as \verb$i += 3;$
\item  Many other operators are possible e.g. \verb$-=, *=, /=$.
\end{itemize}
\end{column}

\pause
\begin{column}{0.5\textwidth}
\lstinputlisting[style=basicc]{../Code/ChapC/power2.c}
\outputlisting{../Code/ChapC/power2.autoout}
\end{column}

\end{columns}
\end{frame}

\begin{frame}[fragile]
\frametitle{The Standard Library}
\begin{columns}

\begin{column}{0.60\textwidth}
\lstinputlisting[style=basicc]{../Code/ChapC/randnums.c}
\outputlisting{../Code/ChapC/randnums.manout}
\end{column}

\pause
\begin{column}{0.35\textwidth}
\begin{itemize}[<+->]
\item Definitions required for the proper use of
many functions such as \verb^rand()^
are found in \verb^stdlib.h^.
\item  Do not mistake these header files for the libraries
themselves~!
\end{itemize}
\end{column}

\end{columns}
\end{frame}
