\section{P : Linked Data Structures}
\label{chap:linked}

%%%%%%%%%%%%%%%%%%%%%%%%%%%%%%%%%%%%%%%%%%%%%%%%%%%%%%%%%%%%%%

\begin{frame}[fragile]
\frametitle{Linked Data Structures}
\begin{columns}[T]

\begin{column}{0.45\textwidth}
\begin{itemize}[<+->]
\item Linked data representations are useful when:
\begin{itemize}[<+->]
\item It is difficult to predict the size and the shape of the data structures in advance.
\item We need to efficiently insert and delete elements.
\end{itemize}
\item To create linked data representations we use pointers to connect separate
blocks of storage together. If a given block contains a pointer to a second
block, we can follow this pointer there.
\item By following pointers one after another, we can travel right along
the structure.
\end{itemize}
\end{column}

\pause
\begin{column}{0.45\textwidth}
\lstinputlisting[style=basicc,linerange={1-34}]{../Code/ChapP/llist.c}
\end{column}

\end{columns}
\end{frame}

%%%%%%%%%%%%%%%%%%%%%%%%%%%%%%%%%%%%%%%%%%%%%%%%%%%%%%%%%%%%%%


\begin{frame}[fragile]
\frametitle{Linked Lists}
\begin{columns}[T]

\begin{column}{0.45\textwidth}
\lstinputlisting[style=basicc,linerange={36-54},numbers=none]{../Code/ChapP/llist.c}
\end{column}

\pause
\begin{column}{0.45\textwidth}
Searching and Recursive printing:
\lstinputlisting[style=basicc,linerange={56-74},numbers=none]{../Code/ChapP/llist.c}
\end{column}

\end{columns}
\end{frame}

%%%%%%%%%%%%%%%%%%%%%%%%%%%%%%%%%%%%%%%%%%%%%%%%%%%%%%%%%%%%%%

\begin{frame}[fragile]
\frametitle{Abstract Data Types}
\begin{columns}[T]

\begin{column}{0.90\textwidth}
\begin{itemize}[<+->]
\item But would we really code something like this {\bf every} time we
need flexible data storage~? 
\item This would be horribly error-prone.
\item Build something once, and test it well.
\item One example of this is an {\bf Abstract Data Type (ADT)}.
\item Each ADT exposes its functionality via an {\em interface}.
The {\bf user} only accesses the data via this interface.
\item The {\bf user} of the ADT doesn't need to understand how the data is
being stored (e.g. array vs. linked lists etc.)
\item Actually, I'll sometimes blur the boundaries of Data Structures (e.g. a linked list) with ADTs (e.g. a dictionary) themselves.
\end{itemize}
\end{column}

\end{columns}
\end{frame}

%%%%%%%%%%%%%%%%%%%%%%%%%%%%%%%%%%%%%%%%%%%%%%%%%%%%%%%%%%%%%%
