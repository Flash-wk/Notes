\section{I : Strings}
\label{chap:str}

%%%%%%%%%%%%%%%%%%%%%%%%%%%%%%%%%%%%%%%%%%%%%%%%%%%%%%%%%%%%%%%%%%

\begin{frame}[fragile]
\frametitle{Strings}
\begin{columns}
\begin{column}{0.45\textwidth}
\begin{itemize}[<+->]
\item Strings are 1D arrays of characters.
\item Any character in a string may be accessed as an array
element.
\item The important difference between strings and ordinary arrays
is the {\bf end-of-string sentinel} \verb^'\0'^ or null character.
\item The string "abc" has a {\it length} of 3, but its {\it size} is 4.
\item Note \verb^'a'^ and \verb^"a"^ are different. The first is a
character constant, the second is a string with 2 elements
\verb^'a'^ and \verb^'\0'^.
\end{itemize}
\end{column}

\pause
\begin{column}{0.45\textwidth}
Initialising Strings~:
\begin{itemize}[<+->]
\item \verb^char w[6] = "Hello";^
\item \begin{verbatim}
char w[250];
w[0] = 'a';
w[1] = 'b';
w[2] = 'c';
w[3] = '\0';
\end{verbatim}
\item \begin{verbatim}
scanf("%s", w);
\end{verbatim}
Removes leading spaces, reads a string (terminated by a
space or \verb^EOF^). Adds a null character to the end
of the string.
\item \begin{verbatim}
char w[250] = {'a', 'b', 'c', '\0'};
\end{verbatim}
\end{itemize}
\end{column}

\end{columns}
\end{frame}

%%%%%%%%%%%%%%%%%%%%%%%%%%%%%%%%%%%%%%%%%%%%%%%%%%%%%%%%%%%%%%%%%%


\begin{frame}[fragile]
\frametitle{Unused Letters and {\tt string.h}}
\begin{columns}

\begin{column}{0.45\textwidth}
\lstinputlisting[style=basicc]{../Code/ChapI/lazydog.c}
\outputlisting{../Code/ChapI/lazydog.autoout}
\end{column}

\pause
\begin{column}{0.45\textwidth}
In \verb^#include <string.h>^ :
{\small
\begin{verbatim}
char *strcat(char dest[], const char src[]);
int strcmp(const char s1[], const char s2[]);
\end{verbatim}
}
\pause
\begin{itemize}[<+->]
\item \verb^strcat()^ appends a copy of string \verb^src^,
including  the  terminating null character,
to  the  end  of  string  \verb^dst^.
\item \verb^strcmp()^ compares two strings byte-by-byte, according to the
     ordering  of  your  machine's  character  set.  The function
     returns an integer greater than, equal to, or less  than  0,
     if the string pointed to by \verb^s1^ is greater than, equal to, or
     less than the string pointed to by \verb^s2^ respectively.
\end{itemize}
\end{column}

\end{columns}
\end{frame}


%%%%%%%%%%%%%%%%%%%%%%%%%%%%%%%%%%%%%%%%%%%%%%%%%%%%%%%%%%%%%%%%%%

\begin{frame}[fragile]
\frametitle{More {\tt string.h}}
\begin{columns}

\begin{column}{0.45\textwidth}
In \verb^#include <string.h>^ :
{\small
\begin{verbatim}
char *strcpy(char dst[], const char src[]);
unsigned strlen(const char s[]);
\end{verbatim}
}
\pause
\begin{itemize}[<+->]
\item \verb^strcpy()^ copies string \verb^src^ to \verb^dst^
including the  terminating null  character,
stopping after the null character has been copied.
\item \verb^strlen()^ returns the number of bytes in \verb^s^,
not including the terminating null character.
\end{itemize}
\end{column}

\pause
\begin{column}{0.45\textwidth}
One way to write the function \verb^strlen()^~:
\lstinputlisting[style=basicc]{../Code/ChapI/strlen.c}
\end{column}

\end{columns}
\end{frame}

%%%%%%%%%%%%%%%%%%%%%%%%%%%%%%%%%%%%%%%%%%%%%%%%%%%%%%%%%%%%%%%%%%

\begin{frame}[fragile]
\begin{columns}

\begin{column}{0.45\textwidth}
In \verb^#include <string.h>^ :
\frametitle{The {\tt snprintf()} Function}
This is very similar to the function \verb^printf()^, except
that the output is stored in a string rather than written
to the output.
It is defined as:
\vspace{-0.75em}
{\small
\begin{verbatim}
int snprintf(string, str-size, control-arg, other args);
\end{verbatim}
}
\pause
\vspace{-0.75em}
For example:
\vspace{-0.75em}
\begin{verbatim}
   int i = 7;
   float f = 17.041;
   char str[100];
   snprintf(str, 100, "%i %f", i, f);
   printf("%s\n", str);

Outputs : 7 17.041000
\end{verbatim}
\vspace{-0.75em}
This is useful if you need to create a string for passing
to another function for further processing.
\end{column}

\pause
\begin{column}{0.45\textwidth}
\lstinputlisting[style=basicc,linerange={74-107},numbers=none]{../Code/ChapI/cards3.c}
\end{column}

\end{columns}
\end{frame}

%%%%%%%%%%%%%%%%%%%%%%%%%%%%%%%%%%%%%%%%%%%%%%%%%%%%%%%%%%%%%%%%%%

\begin{frame}[fragile]
\frametitle{snprintf() and sscanf()}
\begin{columns}

\begin{column}{0.45\textwidth}
\lstinputlisting[style=basicc,linerange={109-130},numbers=none]{../Code/ChapI/cards3.c}
\end{column}

\pause
\begin{column}{0.45\textwidth}
\lstinputlisting[style=basicc,numbers=none]{../Code/ChapI/sscanf.c}
\outputlisting{../Code/ChapI/sscanf.manout}
\end{column}

\end{columns}
\end{frame}
